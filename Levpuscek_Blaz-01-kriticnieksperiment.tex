\DeclareUnicodeCharacter{2212}{-}
\documentclass[a4paper,12pt]{article}
\usepackage[slovene]{babel}
\usepackage[utf8]{inputenc}
\usepackage[T1]{fontenc}
\usepackage{lmodern}
\usepackage{amsmath}
\usepackage{amsfonts} 
\usepackage{tikz}
\usepackage{pgfplots}
\pagestyle{empty} 
\pgfplotsset{compat=1.16}

\begin{document}

\title{Kriitični eksperiment}
\author{Blaž Levpušček}
\date{11. 10. 2021}
\maketitle



\section{Teoretični uvod}
Stanje reaktorja nam opisuje 
pomnoževalni faktor $k$, ki nam 
pove, za kolikokrat se spremeni
število nevtronov v reaktorju iz 
ene generacije v drugo. 
Kadar je $k < 1$ je reaktor 
podkritičen. Nevtronov je vedno 
manj in če v njemu ni neodvisnega 
izvora nevtronov se zaustavi. 
Kadar je $k = 1$ je reaktor kritičen,
število nevtronov in moč reaktorja sta 
konstantna. 
Kadar je $k > 1$ je reaktro nadkritičen, 
moč reaktorja narašča. 
\\
Definiramo še reaktivnost $\rho = \frac{k - 1}{k}$.
Merimo jo lahko v enotah pcm ($\rho[pcm] = \frac{k - 1}{k}\cdot 10^5$). 
Lahko tudi v enotah $\$$ ($\rho[\$] = \frac{k - 1}{k \beta}$), 
kjer je $\beta$ delež zakasnelih nevtronov.
Za reaktor TRIGA ta znaša $\beta = 0,007000 \pm 0,000003$.


\section{Eksperiment}


\section{Meritve}

\begin{tabular}{|c|c|}

    \hline
   t [s] & N \\
   \hline
   60 & 164\\
  
    \hline
\end{tabular}

\section{Analiza in rezultati}


\section{Zaključek}


\end{document}